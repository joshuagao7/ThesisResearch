\begin{abstract}
Over the last decade, athletic training has become increasingly scientific and data-driven in nature. Sports teams commonly incorporate techniques from biomechanics and data science to optimize performance, prevent and predict injuries, and make strategic in-game decisions. Training with sports science involves (1) collecting human performance data, (2) analyzing said data, and (3) translating findings into practical athletic training improvements. This paper seeks to improve upon current techniques for step 2: analyzing human performance data. Effective analysis requires reliably identifying key events to enable long-term comparative analysis. Existing algorithms require user adherence to stringent testing protocols, making in-situ testing processes cumbersome and natural movement data prohibitively challenging to assess. We propose a framework for parsing temporal movement data that mimics human intuition for pattern classification. We use ground reaction force (GRF) data from pressure plates, treating vertical jumps as the target event. We develop several complementary detection algorithms and borrow techniques from machine learning to develop a generalizable, computationally lightweight, linear-time complexity jump detection algorithm. A training dataset of 279 jumps across 27 participants was collected, with the optimal algorithm achieving 28 errors (90\% accuracy). The framework has been successfully deployed in Vault One, a mobile application used by trainers and athletes.
\end{abstract}

